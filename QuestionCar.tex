% Options for packages loaded elsewhere
\PassOptionsToPackage{unicode}{hyperref}
\PassOptionsToPackage{hyphens}{url}
\PassOptionsToPackage{dvipsnames,svgnames,x11names}{xcolor}
%
\documentclass[
]{article}
\usepackage{amsmath,amssymb}
\usepackage{iftex}
\ifPDFTeX
  \usepackage[T1]{fontenc}
  \usepackage[utf8]{inputenc}
  \usepackage{textcomp} % provide euro and other symbols
\else % if luatex or xetex
  \usepackage{unicode-math} % this also loads fontspec
  \defaultfontfeatures{Scale=MatchLowercase}
  \defaultfontfeatures[\rmfamily]{Ligatures=TeX,Scale=1}
\fi
\usepackage{lmodern}
\ifPDFTeX\else
  % xetex/luatex font selection
\fi
% Use upquote if available, for straight quotes in verbatim environments
\IfFileExists{upquote.sty}{\usepackage{upquote}}{}
\IfFileExists{microtype.sty}{% use microtype if available
  \usepackage[]{microtype}
  \UseMicrotypeSet[protrusion]{basicmath} % disable protrusion for tt fonts
}{}
\makeatletter
\@ifundefined{KOMAClassName}{% if non-KOMA class
  \IfFileExists{parskip.sty}{%
    \usepackage{parskip}
  }{% else
    \setlength{\parindent}{0pt}
    \setlength{\parskip}{6pt plus 2pt minus 1pt}}
}{% if KOMA class
  \KOMAoptions{parskip=half}}
\makeatother
\usepackage{xcolor}
\usepackage[margin=1in]{geometry}
\usepackage{color}
\usepackage{fancyvrb}
\newcommand{\VerbBar}{|}
\newcommand{\VERB}{\Verb[commandchars=\\\{\}]}
\DefineVerbatimEnvironment{Highlighting}{Verbatim}{commandchars=\\\{\}}
% Add ',fontsize=\small' for more characters per line
\usepackage{framed}
\definecolor{shadecolor}{RGB}{248,248,248}
\newenvironment{Shaded}{\begin{snugshade}}{\end{snugshade}}
\newcommand{\AlertTok}[1]{\textcolor[rgb]{0.94,0.16,0.16}{#1}}
\newcommand{\AnnotationTok}[1]{\textcolor[rgb]{0.56,0.35,0.01}{\textbf{\textit{#1}}}}
\newcommand{\AttributeTok}[1]{\textcolor[rgb]{0.13,0.29,0.53}{#1}}
\newcommand{\BaseNTok}[1]{\textcolor[rgb]{0.00,0.00,0.81}{#1}}
\newcommand{\BuiltInTok}[1]{#1}
\newcommand{\CharTok}[1]{\textcolor[rgb]{0.31,0.60,0.02}{#1}}
\newcommand{\CommentTok}[1]{\textcolor[rgb]{0.56,0.35,0.01}{\textit{#1}}}
\newcommand{\CommentVarTok}[1]{\textcolor[rgb]{0.56,0.35,0.01}{\textbf{\textit{#1}}}}
\newcommand{\ConstantTok}[1]{\textcolor[rgb]{0.56,0.35,0.01}{#1}}
\newcommand{\ControlFlowTok}[1]{\textcolor[rgb]{0.13,0.29,0.53}{\textbf{#1}}}
\newcommand{\DataTypeTok}[1]{\textcolor[rgb]{0.13,0.29,0.53}{#1}}
\newcommand{\DecValTok}[1]{\textcolor[rgb]{0.00,0.00,0.81}{#1}}
\newcommand{\DocumentationTok}[1]{\textcolor[rgb]{0.56,0.35,0.01}{\textbf{\textit{#1}}}}
\newcommand{\ErrorTok}[1]{\textcolor[rgb]{0.64,0.00,0.00}{\textbf{#1}}}
\newcommand{\ExtensionTok}[1]{#1}
\newcommand{\FloatTok}[1]{\textcolor[rgb]{0.00,0.00,0.81}{#1}}
\newcommand{\FunctionTok}[1]{\textcolor[rgb]{0.13,0.29,0.53}{\textbf{#1}}}
\newcommand{\ImportTok}[1]{#1}
\newcommand{\InformationTok}[1]{\textcolor[rgb]{0.56,0.35,0.01}{\textbf{\textit{#1}}}}
\newcommand{\KeywordTok}[1]{\textcolor[rgb]{0.13,0.29,0.53}{\textbf{#1}}}
\newcommand{\NormalTok}[1]{#1}
\newcommand{\OperatorTok}[1]{\textcolor[rgb]{0.81,0.36,0.00}{\textbf{#1}}}
\newcommand{\OtherTok}[1]{\textcolor[rgb]{0.56,0.35,0.01}{#1}}
\newcommand{\PreprocessorTok}[1]{\textcolor[rgb]{0.56,0.35,0.01}{\textit{#1}}}
\newcommand{\RegionMarkerTok}[1]{#1}
\newcommand{\SpecialCharTok}[1]{\textcolor[rgb]{0.81,0.36,0.00}{\textbf{#1}}}
\newcommand{\SpecialStringTok}[1]{\textcolor[rgb]{0.31,0.60,0.02}{#1}}
\newcommand{\StringTok}[1]{\textcolor[rgb]{0.31,0.60,0.02}{#1}}
\newcommand{\VariableTok}[1]{\textcolor[rgb]{0.00,0.00,0.00}{#1}}
\newcommand{\VerbatimStringTok}[1]{\textcolor[rgb]{0.31,0.60,0.02}{#1}}
\newcommand{\WarningTok}[1]{\textcolor[rgb]{0.56,0.35,0.01}{\textbf{\textit{#1}}}}
\usepackage{graphicx}
\makeatletter
\def\maxwidth{\ifdim\Gin@nat@width>\linewidth\linewidth\else\Gin@nat@width\fi}
\def\maxheight{\ifdim\Gin@nat@height>\textheight\textheight\else\Gin@nat@height\fi}
\makeatother
% Scale images if necessary, so that they will not overflow the page
% margins by default, and it is still possible to overwrite the defaults
% using explicit options in \includegraphics[width, height, ...]{}
\setkeys{Gin}{width=\maxwidth,height=\maxheight,keepaspectratio}
% Set default figure placement to htbp
\makeatletter
\def\fps@figure{htbp}
\makeatother
\setlength{\emergencystretch}{3em} % prevent overfull lines
\providecommand{\tightlist}{%
  \setlength{\itemsep}{0pt}\setlength{\parskip}{0pt}}
\setcounter{secnumdepth}{-\maxdimen} % remove section numbering
\usepackage{color}
\ifLuaTeX
  \usepackage{selnolig}  % disable illegal ligatures
\fi
\IfFileExists{bookmark.sty}{\usepackage{bookmark}}{\usepackage{hyperref}}
\IfFileExists{xurl.sty}{\usepackage{xurl}}{} % add URL line breaks if available
\urlstyle{same}
\hypersetup{
  pdftitle={Assigment 2},
  pdfauthor={Luis Hinostroza},
  colorlinks=true,
  linkcolor={Maroon},
  filecolor={Maroon},
  citecolor={Blue},
  urlcolor={blue},
  pdfcreator={LaTeX via pandoc}}

\title{Assigment 2}
\author{Luis Hinostroza}
\date{2023-10-06}

\begin{document}
\maketitle

\hypertarget{questions-about-cars-from-a-sample-data.}{%
\subsection{Questions about Cars from a Sample
Data.}\label{questions-about-cars-from-a-sample-data.}}

In this I used \ldots{}

\newpage

\hypertarget{q01---does-more-horsepower-decreases-the-acceleration-time}{%
\subsection{Q\#01 - Does more horsepower decreases the acceleration
time?}\label{q01---does-more-horsepower-decreases-the-acceleration-time}}

Looking at the graph we can notice there is a relation between hp and
acceleration. The more hp the less time accelerating. Also the
correlation coefficient represent a pretty strong linear relation.

\begin{Shaded}
\begin{Highlighting}[]
\FunctionTok{library}\NormalTok{(ggplot2)}
\FunctionTok{library}\NormalTok{(ISLR)}
\FunctionTok{library}\NormalTok{(dplyr)}
\NormalTok{hp\_acce }\OtherTok{\textless{}{-}} \FunctionTok{select}\NormalTok{(Auto, horsepower, acceleration)}
\FunctionTok{ggplot}\NormalTok{(hp\_acce, }\FunctionTok{aes}\NormalTok{(}\AttributeTok{x =}\NormalTok{ horsepower, }\AttributeTok{y =}\NormalTok{ acceleration)) }\SpecialCharTok{+}
  \FunctionTok{geom\_point}\NormalTok{() }\SpecialCharTok{+}
  \FunctionTok{geom\_smooth}\NormalTok{(}\AttributeTok{method =} \StringTok{"lm"}\NormalTok{, }\AttributeTok{se =} \ConstantTok{FALSE}\NormalTok{, }\AttributeTok{color =} \StringTok{"red"}\NormalTok{) }\SpecialCharTok{+}
  \FunctionTok{labs}\NormalTok{(}\AttributeTok{title =} \StringTok{"Relationship between Horsepower and Acceleration"}\NormalTok{,}
       \AttributeTok{x =} \StringTok{"Horsepower"}\NormalTok{,}
       \AttributeTok{y =} \StringTok{"Acceleration"}\NormalTok{)}
\end{Highlighting}
\end{Shaded}

\includegraphics{QuestionCar_files/figure-latex/unnamed-chunk-1-1.pdf}

\begin{Shaded}
\begin{Highlighting}[]
\FunctionTok{cor}\NormalTok{(Auto}\SpecialCharTok{$}\NormalTok{horsepower, Auto}\SpecialCharTok{$}\NormalTok{acceleration)}
\end{Highlighting}
\end{Shaded}

\begin{verbatim}
## [1] -0.6891955
\end{verbatim}

\newpage

\hypertarget{q02---does-the-mpg-depends-by-the-amount-of-cylinders}{%
\subsection{Q\#02 - Does the mpg depends by the amount of
cylinders?}\label{q02---does-the-mpg-depends-by-the-amount-of-cylinders}}

The graph shows that the right amount of cylinder can improve the mpg,
Four cylinders been the most efficient miles per gallons. The p-value
from the ANOVA test shows there's a statistically significant difference
in the mean mpg across different cylinders groups.

\begin{Shaded}
\begin{Highlighting}[]
\FunctionTok{library}\NormalTok{(ggplot2)}
\FunctionTok{library}\NormalTok{(ISLR)}
\FunctionTok{library}\NormalTok{(dplyr)}
\NormalTok{mpg\_cyl }\OtherTok{\textless{}{-}} \FunctionTok{select}\NormalTok{(Auto, mpg, cylinders)}
\FunctionTok{ggplot}\NormalTok{(mpg\_cyl, }\FunctionTok{aes}\NormalTok{(}\AttributeTok{x =} \FunctionTok{as.factor}\NormalTok{(cylinders), }\AttributeTok{y =}\NormalTok{ mpg)) }\SpecialCharTok{+}
  \FunctionTok{geom\_boxplot}\NormalTok{() }\SpecialCharTok{+}
  \FunctionTok{labs}\NormalTok{(}\AttributeTok{title =} \StringTok{"Relationship between MPG and Cylinders"}\NormalTok{,}
       \AttributeTok{x =} \StringTok{"Number of Cylinders"}\NormalTok{,}
       \AttributeTok{y =} \StringTok{"MPG"}\NormalTok{) }\SpecialCharTok{+}
  \FunctionTok{theme\_minimal}\NormalTok{()}
\end{Highlighting}
\end{Shaded}

\includegraphics{QuestionCar_files/figure-latex/unnamed-chunk-2-1.pdf}

\begin{Shaded}
\begin{Highlighting}[]
\NormalTok{anova\_result }\OtherTok{\textless{}{-}} \FunctionTok{aov}\NormalTok{(mpg }\SpecialCharTok{\textasciitilde{}} \FunctionTok{as.factor}\NormalTok{(cylinders), }\AttributeTok{data =}\NormalTok{ Auto)}
\FunctionTok{summary}\NormalTok{(anova\_result)}
\end{Highlighting}
\end{Shaded}

\begin{verbatim}
##                       Df Sum Sq Mean Sq F value Pr(>F)    
## as.factor(cylinders)   4  15275    3819     173 <2e-16 ***
## Residuals            387   8544      22                   
## ---
## Signif. codes:  0 '***' 0.001 '**' 0.01 '*' 0.05 '.' 0.1 ' ' 1
\end{verbatim}

\newpage

\hypertarget{q03---is-the-amount-of-cylinders-related-to-the-origin}{%
\subsection{Q\#03 - Is the amount of cylinders related to the
origin?}\label{q03---is-the-amount-of-cylinders-related-to-the-origin}}

We can notice from the graph the distribution of car by number of
cylinders. Orange being form USA, pink from EU, and red from Japan. The
p-value indicates that the two variables are not independent and that
there is a significant association between cylinders and origin.

\begin{Shaded}
\begin{Highlighting}[]
\FunctionTok{library}\NormalTok{(ggplot2)}
\FunctionTok{library}\NormalTok{(ISLR)}
\FunctionTok{library}\NormalTok{(dplyr)}
\FunctionTok{library}\NormalTok{(vcd)}
\NormalTok{table\_cyl\_origin }\OtherTok{\textless{}{-}} \FunctionTok{table}\NormalTok{(Auto}\SpecialCharTok{$}\NormalTok{cylinders, Auto}\SpecialCharTok{$}\NormalTok{origin)}
\FunctionTok{print}\NormalTok{(table\_cyl\_origin)}
\end{Highlighting}
\end{Shaded}

\begin{verbatim}
##    
##       1   2   3
##   3   0   0   4
##   4  69  61  69
##   5   0   3   0
##   6  73   4   6
##   8 103   0   0
\end{verbatim}

\begin{Shaded}
\begin{Highlighting}[]
\FunctionTok{mosaicplot}\NormalTok{(table\_cyl\_origin, }\AttributeTok{main=}\StringTok{"Mosaic Plot of Cylinders and Origin"}\NormalTok{, }\AttributeTok{xlab=}\StringTok{"Cylinders"}\NormalTok{, }\AttributeTok{ylab=}\StringTok{"Origin"}\NormalTok{, }\AttributeTok{col=}\FunctionTok{c}\NormalTok{(}\StringTok{"orange"}\NormalTok{, }\StringTok{"pink"}\NormalTok{, }\StringTok{"red"}\NormalTok{))}
\end{Highlighting}
\end{Shaded}

\includegraphics{QuestionCar_files/figure-latex/unnamed-chunk-3-1.pdf}

\begin{Shaded}
\begin{Highlighting}[]
\NormalTok{chi\_sq\_test }\OtherTok{\textless{}{-}} \FunctionTok{chisq.test}\NormalTok{(table\_cyl\_origin)}
\FunctionTok{print}\NormalTok{(chi\_sq\_test)}
\end{Highlighting}
\end{Shaded}

\begin{verbatim}
## 
##  Pearson's Chi-squared test
## 
## data:  table_cyl_origin
## X-squared = 180.72, df = 8, p-value < 2.2e-16
\end{verbatim}

\end{document}
